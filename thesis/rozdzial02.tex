\chapter{Internetowe strumienie wideo}

Jakieś wprowadzenie o wideo ogólnie, w jakich kontekstach występuje (pliki wideo, streaming np. Youtube/Netflix,
strumienie czasu rzeczywistego np. Twitch/Discord/Teams), czym te konteksty się różnią.

Zidealizowany obraz rozmowy wideo w internecie:
\begin{enumerate}
	\item Komputery są publicznymi hostami w internecie i program komunikatora słucha na danym porcie
	\item Strona nawiązująca połączenie łączy się do hosta odbiorcy po tym porcie, sygnalizuje chęć nawiązania
	      połączenia
	\item Strona odbierająca akceptuje
	\item Kamera nadawcy przechwytuje najlepszy obraz jaki może, klatka po klatce i przesyła go do komputera
	\item Komputer wysyła przechwycone klatki wcześniej ustanowionym kanałem
\end{enumerate}

Natomiast pojawiają się problemy:

\begin{enumerate}
	\item Komputery znajdują się w sieciach domowych, za NATem, nie można się do nich bezpośrednio połączyć
	\item Nieskompresowane wideo jest zbyt duże by wysłać je przez internet, potrzebny jest jakiś mechanizm kompresji
	\item Komputery mogą mieć różne możliwości przetwarzania wideo: znajdować się w sieciach o znacząco różnej
	      szybkości, mieć różniące się szybkością procesory, kamery zapisujące klatki w różnych formatach
\end{enumerate}
