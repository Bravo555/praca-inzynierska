\chapter{Internetowe strumienie wideo}

Jakieś wprowadzenie o wideo ogólnie, w jakich kontekstach występuje (pliki wideo, streaming np. Youtube/Netflix,
strumienie czasu rzeczywistego np. Twitch/Discord/Teams), czym te konteksty się różnią.

Możemy wyróżnić 3 rodzaje wideo:

\begin{itemize}
	\item lokalne pliki wideo - pliki wideo na dysku w kontenerze mp4, mkv, lub innym
	\item streamowanie plików wideo (np. Youtube albo Netflix) - mamy przygotowane segmenty pliku wideo w różnych
	      rozdzielczościach i wysyłamy je po kolei, dobieramy rozdzielczość wg. dostępnego pasma pomiędzy serwerem a
	      klientem
	\item strumieniowanie w czasie rzeczywistym - priorytetem jest opóźnienie, kodujemy na bieżąco przechwytywane klatki
	      i wysyłamy je najszybciej jak się da; przez to niektóre mechanizmy kompresji są niedostępne (np. B-klatki,
	      które wykorzystują dane z następnej klatki aby zmniejszyć wielkość klatki)
\end{itemize}

Zidealizowany obraz rozmowy wideo w internecie:
\begin{enumerate}
	\item Komputery są publicznymi hostami w internecie i program komunikatora słucha na danym porcie
	\item Strona nawiązująca połączenie łączy się do hosta odbiorcy po tym porcie, sygnalizuje chęć nawiązania
	      połączenia
	\item Strona odbierająca akceptuje
	\item Kamera oraz mikrofon nadawcy przechwytują najlepszy możliwy obraz i dzwięk, i przesyłają je do komputera
	\item Strumienie wideo i audio są łączone i synchronizowane
	\item Komputer wysyła strumień audio-wideo wcześniej ustanowionym kanałem
\end{enumerate}

Natomiast pojawiają się problemy:

\begin{enumerate}
	\item Komputery znajdują się w sieciach domowych, za NATem, nie można się do nich bezpośrednio połączyć
	\item Nieskompresowane wideo jest zbyt duże by wysłać je przez internet, potrzebny jest jakiś mechanizm kompresji
	\item Komputery mogą mieć różne możliwości przetwarzania wideo: znajdować się w sieciach o znacząco różnej
	      szybkości, mieć różniące się szybkością procesory, kamery zapisujące klatki w różnych formatach
\end{enumerate}
