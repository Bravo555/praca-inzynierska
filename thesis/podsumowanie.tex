\chapter{Podsumowanie}
\label{chap:podsumowanie}
Podsumowanie jest miejscem, w którym należy zamieścić syntetyczny opis tego, o czym jest dokument. W
szczególności w pracach dyplomowych w podsumowaniu powinno znaleźć się jawnie podane stwierdzenie
dotyczące stopnia realizacji celu. Czyli powinny pojawić się w niej akapity ze zdaniami typu:
,,Podczas realizacji pracy udało się zrealizować wszystkie postawione cele''. Ponadto powinna
pojawić się dyskusja na temat napotkanych przeszkód i sposobów ich pokonania, perspektyw dalszego
rozwoju, możliwych zastosowań wyników pracy itp.

\section{Realizacja celu pracy}

Zadanie utworzenia komunikatora prezentującego reguły prowadzenia połączeń peer-to-peer zostały
zrealizowane.

\section{Wnioski}

Wykonany komunikator nie jest jednak gotowy do wykorzystania w środowisku produkcyjnym
ponieważ operuje on na zasadzie globalnej widoczności wszystkich połączonych użytkowników, którzy
identyfikowani są tylko po nazwie, którą każdy z użytkowników może wybrać dowolnie podczas
dołączania, o ile nie jest ona już zajęta.

Mimo tego, komunikator obrazuje jak łatwe może być tworzenie multimedialnych aplikacji
komunikujących się w architekturze peer-to-peer. Gwarancje bezpieczeństwa języka Rust sprawiają że
deweloper może zapomnieć o segfaultach i memory corruption, a ekspresywny system typów sprawia że
powstały kod wygląda jak kod napisany w języku wysokopoziomowym takim jak np. Java, jednocześnie
będąc szybkim jak kod w języku niskopoziomowym (np. C) dzięki abstrakcjom o zerowych kosztach.

Elementami który dodały najwięcej złożoności i potencjału na wprowadzenie błędów były GStreamer oraz
GTK. Są one napisane w C, i ich bindingi w języku Rust są tylko obudową na ogrom kodu C, który
replikuje i zastępuje wiele funkcjonalności wykonywanych lepiej przez Rusta.

\section{Możliwe usprawnienia}

\begin{itemize}
    \item Wyświetlanie statusu użytkowników na liście wszystkich użytkowników
    \item Kontrolki do włączenia/wyłączenia kamerki/mikrofonu w oknie połączenia
    \item Wykorzystanie pełnoprawnego frameworka architektury aktorów, np. \href{https://github.com/actix/actix}{Actix}.
\end{itemize}


%\show\chapter
%\show\section
%\show\subsection

%\showthe\secindent
%\showthe\beforesecskip
%\showthe\aftersecskip
%\showthe\secheadstyle
%\showthe\subsecindent
%\showthe\beforesubsecskip
%\showthe\aftersubsecskip
%\showthe\subseccheadstyle
%\showthe\parskip
