\chapter{Podsumowanie}
\label{chap:podsumowanie}
\section{Realizacja celu pracy}

Zadanie utworzenia komunikatora prezentującego reguły prowadzenia połączeń peer-to-peer zostały
zrealizowane. Wykonany komunikator realizuje postanowione wymagane wymagania funkcjonalne, jednak
nie wykorzystuje kodeka AV1, ponieważ dostępne w GStreamer enkodery AV1 okazały się jeszcze
niewystarczająco szybkie do kodowania wideo w czasie rzeczywistym. Modularna architektura frameworka
GStreamer pozwoliła na bezproblemowe zastąpienie AV1 innym nowoczesnym kodekiem - VP8. W miarę
rozwoju enkoderów programowych a także coraz większej dostępności enkoderów sprzętowych w
nowoczesnych układach graficznych, użycie AV1 w komunikatorach wideo może jeszcze stać się
praktyczne w niedalekiej przyszłości.

\section{Wnioski}

Wykonany komunikator nie jest jednak gotowy do wykorzystania w środowisku produkcyjnym
ponieważ operuje on na zasadzie globalnej widoczności wszystkich połączonych użytkowników, którzy
identyfikowani są tylko po nazwie, którą każdy z użytkowników może wybrać dowolnie podczas
dołączania, o ile nie jest ona już zajęta.

Mimo tego, komunikator obrazuje jak łatwe może być tworzenie multimedialnych aplikacji
komunikujących się w architekturze peer-to-peer. Gwarancje bezpieczeństwa języka Rust sprawiają że
deweloper może zapomnieć o segfaultach i memory corruption, a ekspresywny system typów sprawia że
powstały kod wygląda jak kod napisany w języku wysokopoziomowym takim jak np. Java, jednocześnie
będąc szybkim jak kod w języku niskopoziomowym (np. C) dzięki abstrakcjom o zerowych kosztach.

Elementami który dodały najwięcej złożoności i potencjału na wprowadzenie błędów były GStreamer oraz
GTK. Są one napisane w C, i ich bindingi w języku Rust są tylko obudową na ogrom kodu C, który
replikuje i zastępuje wiele funkcjonalności wykonywanych lepiej przez Rusta.

Ponadto, możliwe jest lepsze wykorzystanie systemu typów języka Rust do modelowania stanu aplikacji
tak, by niepoprawne stany były nieosiągalne. Gdyby zaczynać projekt od nowa, autor na początku
dążyłby do wykonania kompletnego modelu celem poprawnej separacji warstw.

\section{Możliwe usprawnienia}

\begin{itemize}
    \item Wyświetlanie statusu użytkowników na liście wszystkich użytkowników
    \item Kontrolki do włączenia/wyłączenia kamerki/mikrofonu w oknie połączenia
    \item Wykorzystanie pełnoprawnego frameworka architektury aktorów, np. \href{https://github.com/actix/actix}{Actix}.
\end{itemize}


%\show\chapter
%\show\section
%\show\subsection

%\showthe\secindent
%\showthe\beforesecskip
%\showthe\aftersecskip
%\showthe\secheadstyle
%\showthe\subsecindent
%\showthe\beforesubsecskip
%\showthe\aftersubsecskip
%\showthe\subseccheadstyle
%\showthe\parskip
