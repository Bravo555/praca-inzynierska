\pdfbookmark[0]{Streszczenie}{streszczenie.1}
%\phantomsection
%\addcontentsline{toc}{chapter}{Streszczenie}
%%% Poniższe zostało niewykorzystane (tj. zrezygnowano z utworzenia nienumerowanego rozdziału na abstrakt)
%%%\begingroup
%%%\setlength\beforechapskip{48pt} % z jakiegoś powodu była maleńka różnica w położeniu nagłówka rozdziału numerowanego i nienumerowanego
%%%\chapter*{\centering Abstrakt}
%%%\endgroup
%%%\label{sec:abstrakt}
%%%Lorem ipsum dolor sit amet eleifend et, congue arcu. Morbi tellus sit amet, massa. Vivamus est id risus. Sed sit amet, libero. Aenean ac ipsum. Mauris vel lectus.
%%%
%%%Nam id nulla a adipiscing tortor, dictum ut, lobortis urna. Donec non dui. Cras tempus orci ipsum, molestie quis, lacinia varius nunc, rhoncus purus, consectetuer congue risus.
%\mbox{}\vspace{2cm} % można przesunąć, w zależności od długości streszczenia
\begin{abstract}
    W ostatnich latach członkowie konsorcjum AOM wprowadzają na rynek nowy standard kompresji wideo
    - AV1. W niektórych zastosowaniach kodek AV1 jest w stanie osiągnąć taką samą jakość wizualną
    jak używany od 2003 standard AVC, zajmując przy tym zaledwie połowę miejsca. Nowy standard ma
    zatem duże znaczenie dla członków konsorcjum, których działalność opiera się w znacznej części
    na strumieniowaniu dużej ilości wideo do wielu klientów, takich jak Google, Netflix, lub Amazon,
    którzy grają pierwsze skrzypce w jego adopcji. Niestety, ponieważ implementacje koderów AV1
    pozostają znacznie wolniejsze od już przyjętych kodeków, aplikacje strumieniujące wideo w czasie
    rzeczywistym, takie jak komunikatory internetowe lub platformy wideokonferencyjne, nie miały
    jeszcze okazji skorzystać z nowego kodeka aby usprawnić jakość transmitowanego wideo.
    
    Tematem pracy jest zbadanie praktyczności wykorzystania nowoczesnego kodeka wideo AV1 w
    komunikatorze wideo. Wykonano w tym celu prosty komunikator internetowy składający się z serwera
    oraz aplikacji graficznej na systemy Linux wykonanej w języku programowania Rust z użyciem
    frameworków GTK oraz GStreamer. Niestety dostępne w GStreamer implementacje kodeków nie były w
    stanie osiągnąć prędkości kodowania pozwalającej na strumieniowanie wideo w czasie rzeczywistym.
    Jednak w miarę rozwoju enkoderów programowych a także coraz większej dostępności enkoderów
    sprzętowych w nowoczesnych układach graficznych, użycie AV1 w komunikatorach wideo może jeszcze
    stać się praktyczne w niedalekiej przyszłości.
\end{abstract}
\mykeywords{AV1}

% Dobrze byłoby skopiować słowa kluczowe do metadanych dokumentu pdf (w pliku Dyplom.tex)
% Niestety, zaimplementowane makro nie robi tego z automatu, więc pozostaje kopiowanie ręczne.

{
    \selectlanguage{english}
    \begin{abstract}
        In recent years the members of the AOM consortium have been rolling out the new video
        compression standard - AV1. The new standard can maintain the same visual quality while
        taking up as little as half of the space compared to AVC, an older standard used from 2003.
        As such, fast adoption of the new standard was pursued by the members of the consortium,
        whose business largely consists of providing video streaming services for many customers,
        such as Google, Netflix, or Amazon. Unfortunately because recent AV1 production codecs
        sometimes still struggle with encoding speed when compared to the older, more established
        codecs, real-time applications like video-chat or videoconferencing applications were not
        yet able to take advantage of the new standard to provide a better quality video streams for
        its users.
        
        This paper aims to examine the feasibility of the recently developed AV1 standard when used
        in a video-chat application. For this purpose, such application was developed using the Rust
        programming language, consisting of a server and a GUI application for Linux systems
        utilizing GTK and GStreamer frameworks. Unfortunately available software codecs were not
        able to achieve a real-time encoding speed. As software codecs improve and hardware codecs
        continue to appear in more products, the usage of AV1 in real-time applications may
        nonetheless become feasible in a near future.
    \end{abstract}
    \mykeywords{AV1}
}
