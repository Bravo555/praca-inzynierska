\chapter{Wstęp}
\section{Wprowadzenie}

Pandemia COVID-19 oraz zdobywające coraz większą popularność zdalne formy zatrudnienia obrazują jak ważna jest rola
połączeń wideo we współczesnym społeczeństwie. W wielu przypadkach kontakt "twarzą w twarz" jest preferowalny, a nawet
niezbędny do realizacji pewnych zadań. W takich wypadkach niezawodność oraz efektywność transmisji wideo stają się
bardzo ważnymi problemami.

Celem niniejszej pracy jest analiza połączeń wideo czasu rzeczywistego w każdym ich etapie, badanie procesów
składających się na nie, i wreszcie utworzenie internetowego komunikatora wideo wykorzystującego poznane koncepty i
rozwiązania.

W rozdziale 2 omówione zostaną koncepty i metody związane ze strumieniowaniem wideo. Przedstawiony zostanie uproszczony
opis procesu transmisji wideo w czasie rzeczywistym, od pobrania klatki obrazu przez kamerę internetową, do wyświetlenia
tejże klatki na monitorze rozmówcy. Poruszone zostaną problemy związane z transmisją wideo przez sieć internetową, np.
problem ustanowienia kanału P2P pomiędzy klientami za siecią NAT, negocjowanie sesji pomiędzy klientami, adaptive
bitrate, etc.

W rozdziale 3 omówiony zostanie projekt WebRTC i jego protokoły składowe, czytelnik dowie się również w jaki sposób
WebRTC rozwiązuje problemy poruszone we wcześniejszym rozdziale i tym samym niezwykle upraszcza tworzenie
multimedialnych aplikacji webowych.

W rozdziale 4 zaprezentowana zostanie aplikacja webowa wykorzystująca WebRTC, nastąpi ogólny przegląd wykorzystywanych
technologii oraz oprogramowania, zaprezentowane zostaną fragmenty kodu źródłowego realizujące kluczowe procesy
nawiązywania połączenia poruszone we wcześniejszym rozdziale. Wykonana zostanie dokładna analiza ruchu sieciowego
pomiędzy hostami i uzupełniony zostanie proces nawiązywania połączenia WebRTC poruszony we wcześniejszym rozdziale,
zobrazowany konkretnym przykładem.

W rozdziale 5 omówione zostaną problemy, generalne mechanizmy i koncepty związane z kompresją wideo, kontenery,
strumienie oraz ich muxowanie do kontenerów, rodzaje kodeków wideo, proces enkodowania/dekodowania, sprzętowe oraz
programowe implementacje koderów.

W rozdziale 6 poruszone zostaną problemy i rozwiązania związane ze strumieniowaniem AV1. Zrealizowane zostanie "zejście
na niższy poziom", co pozwoli na głębsze zapoznanie się z krokami realizowanymi celem przygotowania strumienia wideo do
transmisji oraz wyświetlenia przychodzącego wideo na ekranie monitora. Zrealizowane zostanie min.: zastąpienie całości
lub części stosu WebRTC rozwiązaniami natywnymi, oferowanymi przez sprzęt lub system operacyjny, wyeliminowanie
abstrakcji oferowanych przez przeglądarkę, wybór optymalnego kodera oraz jego parametrów, wybór technologii GUI,
zaplanowanie i omówienie procesu strumieniowania z rozdziału 1szego, wraz z elementami które będą go realizowały.

W rozdziale 7 zaprezentowana zostanie wykonana aplikacja desktopowa wykorzystująca biblioteki oraz inne rozwiązania
udostępniane przez system operacyjny. Omówienie architektury, wykorzystywanych technologii i fragmentów kodu aplikacji
okienkowej. Zaprezentowanie wybranych etapów procesu strumieniowania realizowanych przez aplikację.
