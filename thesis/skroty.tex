\pdfbookmark[0]{Skróty}{skroty.1}% 
%%\phantomsection
%%\addcontentsline{toc}{chapter}{Skróty}
\chapter*{Skróty}
\label{sec:skroty}
\noindent\vspace{-\topsep-\partopsep-\parsep} % Jeśli zaczyna się od otoczenia description, to otoczenie to ląduje lekko niżej niż wylądowałby zwykły tekst, dlatego wstawiano przesunięcie w pionie
\begin{description}[labelwidth=*]
  \item [WebRTC] (ang.\ \emph{Web Real Time Communications })
  \item [ICE] (ang.\ \emph{Interactive Connectivity Establishment})
  \item [STUN] (ang.\ \emph{Session Traversal Utilities for NAT})
  \item [TURN] (ang.\ \emph{Traversal Using Relays around NAT})
\end{description}
