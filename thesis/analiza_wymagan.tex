\chapter{Analiza wymagań}

\section{Wymagania projektowe}

Końcowym celem projektu jest poznanie procesów kluczowych w internetowych transmisjach audio-wideo
oraz wykonanie aplikacji okienkowej na systemy Linux pełniącej rolę komunikatora internetowego.
Komunikator powinien pozwalać użytkownikom na odkrywanie innych użytkowników i prowadzenie z nimi
połączeń audio-wideo. Połączenia audio-wideo pomiędzy użytkownikami będą odbywać się w trybie
peer-to-peer, tj. transmisje wideo i audio w tych połączeniach trafiają na drugą stronę połączenia
bezpośrednio, bez pośrednictwa serwera, który będzie służył tylko i wyłącznie do dwóch celów: by
umożliwiać użytkownikom odkrywanie innych dostępnych użytkowników do których można wykonać
połączenie, oraz jako mechanizm początkowej wymiany danych pomiędzy stronami połączenia celem
sygnalizacji połączenia oraz późniejszego ustanowienia bezpośredniego kanału wymiany danych
peer-to-peer.

\section{Wymagania funkcjonalne}

\begin{enumerate}
	\item Użytkownik może wybrać imię pod którym widoczny będzie dla innych użytkowników
	\item Użytkownik widzi listę aktualnie dostępnych użytkowników
	\item Użytkownik X może zadzwonić do użytkownika Y
	\item Użytkownik X w trakcie oczekiwania na odpowiedź od użytkownika Y, może zrezygnować z
	      połączenia, rozłączyć się
	\item Użytkownik Y, gdy dzwoni do niego użytkownik X, może odrzucić lub odebrać połączenie
	\item Jeśli połączenie zostanie odebrane i poprawnie nawiązane, każdy z użytkowników powinien
	      móc zaobserwować transmisję z kamery wideo oraz usłyszeć transmisję audio z mikrofonu
	      drugiego użytkownika
	\item W trakcie trwania połączenia, każdy z użytkowników może rozłączyć się, unilateralnie
	      terminując połączenie.
	\item (opcjonalne) Użytkownik może zmienić swoją nazwę pod którą jest widoczny bez rozłączania
	      się z serwerem
\end{enumerate}

\section{Wymagania niefunkcjonalne}

\begin{enumerate}
	\item Do realizacji projektu zostanie wykorzystany język programowania Rust
	\item Aplikacja wykorzystuje frameworki i biblioteki natywne dla systemów Linux
	\item Aplikacja powinna działać nie tylko w sieci lokalnej, ale także w sieci Internet
	\item Połączenia pomiędzy użytkownikami powinny odbywać się w trybie peer-to-peer celem
	      minimalizacji opóźnień
	\item Z powodu powyższego wymagania, połączenia są ograniczone do dwóch użytkowników, tj. nie ma
	      możliwości tworzenia konferencji z wieloma użytkownikami.
	\item (opcjonalne) Aplikacja wykorzystuje kodek AV1 do kompresji wideo
	\item (opcjonalne) Aplikacja powinna wdzięcznie obsługiwać brak kamery wideo lub mikrofonu przez
	      użytkownika
\end{enumerate}
